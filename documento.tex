\documentclass[12pt,a4paper]{article}
\usepackage[utf8]{inputenc}
\usepackage[brazil]{babel}
\usepackage{array}
\usepackage{geometry}
\usepackage{graphicx}
\usepackage{longtable}
\usepackage{enumitem}

\geometry{margin=2.5cm}

\title{\textbf{DCC174 - Interação Humano-Computador \\ Parte I – Requisitos de IHC}}
\author{Frederico Dôndici Gama Vieira \\ Rafael de Oliveira Zimbrão \\ João Vitor Fernandes Ribeiro Carneiro Ramos \\ Giovanni Almeida Dutra}
\date{Juiz de Fora, Maio 2025}

\begin{document}

\maketitle

\section{Análise da Situação Atual}

\subsection{Perfis de Usuário}

Foram identificados dois perfis principais: Passageiro e Motorista.

\begin{longtable}{|p{4cm}|p{5cm}|p{5cm}|}
\hline
\textbf{Característica} & \textbf{Passageiro} & \textbf{Motorista} \\
\hline
Faixa Etária & [18, 60) & [18, 60) \\
\hline
Ocupação & Estudantes, trabalhadores assalariados, autônomos. & Motorista de aplicativo (integral ou complementar). \\
\hline
Educação & Diversificado. & Nível fundamental ou médio. \\
\hline
Afinidade com Tecnologia & Média a alta. & Média. \\
\hline
Experiência com Apps Similares & Alta. & Alta. \\
\hline
Tarefas Primárias & Solicitar corrida, ver preço, acompanhar trajeto, pagar. & Aceitar corridas, navegar, finalizar viagem, receber pagamento. \\
\hline
Tarefas Secundárias & Avaliar motorista, ver histórico, contatar suporte. & Avaliar passageiro, comunicar-se com passageiro. \\
\hline
Domínio do Produto & Aprendizado rápido e mínimo. & Aprendizado rápido e mínimo. \\
\hline
Tecnologia Disponível & Smartphones. & Smartphones. \\
\hline
\end{longtable}

\subsection{Caracterização de Personas}

\textbf{Persona 1: Ana Silva, a Estudante Conectada}
\begin{itemize}
    \item \textbf{Identidade:} Ana Silva, 22 anos, estudante de Arquitetura na UFJF.
    \item \textbf{Narrativa:} Mora no bairro São Mateus, utiliza apps de transporte com frequência. Busca custo-benefício, frustra-se com preços dinâmicos e tempo de espera. Vê o Carona PJF como alternativa mais justa.
\end{itemize}

\textbf{Persona 2: Carlos Santos, o Motorista Experiente}
\begin{itemize}
    \item \textbf{Identidade:} Carlos Santos, 45 anos, motorista profissional.
    \item \textbf{Narrativa:} Trabalha em tempo integral como motorista. Queixa-se das altas taxas de comissão. Um app sem taxas é mais atrativo.
\end{itemize}

\subsection{Metas de Design}
\begin{itemize}
    \item Simplicidade e Usabilidade: Interface intuitiva para passageiros e motoristas.
    \item Eficiência: Tempo mínimo entre solicitação e início da corrida.
    \item Segurança: Proteção de dados pessoais e financeiros.
    \item Confiabilidade: Sistema estável com rastreamento preciso.
    \item Comunicabilidade: Clareza sobre funcionalidades, próximos passos e estado do sistema.
\end{itemize}

\section{Síntese da Intervenção}

\subsection{Design da Interação}

\subsubsection{Cenário 1: Solicitação de uma Corrida (Ana Silva)}
Narrativa do uso, seguida por diálogo estruturado e tabela de signos.

\begin{longtable}{|p{1.5cm}|p{2.5cm}|p{5cm}|p{5cm}|}
\hline
\textbf{ID} & \textbf{Tipo} & \textbf{Significado (Conteúdo)} & \textbf{Expressão (Forma)} \\
\hline
S-01 & Conteúdo & Informar destino & Caixa de texto com ícone de lupa \\
\hline
S-02 & Conteúdo & Aceitação do trajeto e valor & Botão verde com texto branco \\
\hline
S-03 & Conteúdo & Trajeto do motorista & Mapa com ícone de carro \\
\hline
\end{longtable}

\subsubsection{Cenário 2: Aceitação e Realização da Corrida (Carlos Santos)}

\begin{longtable}{|p{1.5cm}|p{2.5cm}|p{5cm}|p{5cm}|}
\hline
\textbf{ID} & \textbf{Tipo} & \textbf{Significado (Conteúdo)} & \textbf{Expressão (Forma)} \\
\hline
S-04 & Conteúdo & Detalhes da corrida & Painel com dados da corrida \\
\hline
S-05 & Conteúdo & Confirmação da corrida & Barra horizontal deslizante \\
\hline
S-06 & Conteúdo & Início do trajeto & Botão destacado no app \\
\hline
\end{longtable}

\subsubsection{Cenário 3: Cadastro de um Novo Motorista}

\begin{longtable}{|p{1.5cm}|p{2.5cm}|p{5cm}|p{5cm}|}
\hline
\textbf{ID} & \textbf{Tipo} & \textbf{Significado (Conteúdo)} & \textbf{Expressão (Forma)} \\
\hline
S-07 & Conteúdo & Etapa atual do cadastro & Barra ou série de passos visuais no topo da tela \\
\hline
S-08 & Conteúdo & Envio de documento & Botão com ícone de câmera e texto \\
\hline
S-09 & Conteúdo & Confirmação de envio & Mensagem de sucesso com ícone de verificação \\
\hline
\end{longtable}

\subsection{Análise dos Cenários}
\begin{itemize}
    \item No Cenário 1, clareza no botão de confirmação.
    \item No Cenário 2, card hierarquizado para leitura rápida.
    \item No Cenário 3, comunicação clara para dados sensíveis.
\end{itemize}

\subsection{Mapa de Objetivos dos Usuários}
\textbf{Passageiro:} Final – Chegar ao destino. Diretos – Solicitar corrida, acompanhar motorista, pagar. Indiretos – Cadastro, avaliação.
\textbf{Motorista:} Final – Obter renda. Diretos – Aceitar, realizar corrida, receber pagamento. Indiretos – Cadastro, histórico.

\end{document}
