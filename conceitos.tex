\documentclass[12pt,a4paper]{article}
\usepackage[utf8]{inputenc}
\usepackage[brazil]{babel}
\usepackage{array}
\usepackage{geometry}
\usepackage{longtable}

\geometry{margin=2.5cm}

\title{\textbf{Engenharia Semiótica na Interação Humano–Computador}}
\author{}
\date{}

\begin{document}

\maketitle

\section{Conceito Geral}
A Engenharia Semiótica é uma abordagem teórica e prática da IHC que vê o design de interfaces como um ato de comunicação entre o \textbf{designer} (quem projetou o sistema) e o \textbf{usuário} (quem o utiliza).

Segundo essa perspectiva, o sistema interativo é um ``mensageiro'' que carrega, através de sua interface, uma mensagem duradoura do designer para o usuário, explicando:
\begin{itemize}
    \item Quem o usuário é (segundo a visão do designer);
    \item Quais são as tarefas e objetivos do sistema;
    \item Como interagir para atingir esses objetivos;
    \item O que o sistema pode ou não fazer.
\end{itemize}

\section{Origem e Fundamentação}
\begin{itemize}
    \item Criada e desenvolvida principalmente por Clarisse de Souza e colaboradores no Brasil, na PUC-Rio, a partir do final dos anos 1990.
    \item Baseia-se na \textbf{Semiótica} (estudo dos signos e significados) aplicada ao design de sistemas computacionais.
    \item Enxerga a interface como um \textbf{artefato semiótico}: cada elemento (botão, menu, mensagem, fluxo) transmite significado.
\end{itemize}

\section{Metáfora da Conversa}
A Engenharia Semiótica descreve a interação como se fosse uma conversa mediada:
\begin{enumerate}
    \item Designer $\rightarrow$ Sistema: o designer ``programa'' o sistema para falar em seu nome.
    \item Sistema $\rightarrow$ Usuário: o sistema transmite mensagens sobre como deve ser usado.
    \item Usuário $\rightarrow$ Sistema: o usuário interpreta a mensagem e age.
    \item Feedback: o sistema responde, reforçando ou corrigindo a interpretação do usuário.
\end{enumerate}
Essa mensagem não é dinâmica como uma conversa humana — ela é fixa no momento em que o sistema é projetado, mas deve ser suficiente e clara para guiar o usuário em diversos contextos.

\section{Objetivos Principais}
\begin{itemize}
    \item Tornar explícita a visão do designer sobre:
    \begin{itemize}
        \item Quem é o usuário;
        \item Quais problemas ele quer resolver;
        \item Como o sistema foi pensado para ajudá-lo.
    \end{itemize}
    \item Avaliar a comunicabilidade do sistema — se a mensagem do designer chega de forma clara ao usuário.
    \item Aprimorar a usabilidade e a experiência não só do ponto de vista técnico, mas também comunicacional.
\end{itemize}

\section{Conceitos-Chave}
\begin{itemize}
    \item \textbf{Designer}: emissor original da mensagem.
    \item \textbf{Usuário}: receptor e intérprete da mensagem.
    \item \textbf{Interface}: canal e meio semiótico da comunicação.
    \item \textbf{Mensagem do Designer}: tudo o que o sistema diz sobre seu propósito, uso e funcionamento.
    \item \textbf{Quebra de comunicação}: quando a mensagem não é entendida corretamente, levando a erros, confusão ou frustração.
\end{itemize}

\section{Métodos e Ferramentas}
A Engenharia Semiótica propõe métodos formais para projetar e avaliar a comunicação em sistemas:
\begin{itemize}
    \item \textbf{MoLIC} (Model of Language for Interaction as Conversation): linguagem/modelo para representar a interação como diálogos, destacando objetivos, passos e mensagens.
    \item \textbf{Inspeção de Comunicabilidade}: método de avaliação em que avaliadores identificam possíveis problemas de comunicação na interface, usando categorias como:
    \begin{itemize}
        \item ``O que é isso?''
        \item ``E agora?''
        \item ``Por que isso aconteceu?''
        \item ``Como faço para...?''
        \item ``Onde estou?''
    \end{itemize}
\end{itemize}

\section{Aplicações}
\begin{itemize}
    \item Projetos de novos sistemas: criar interfaces mais claras e consistentes.
    \item Avaliação de sistemas existentes: detectar falhas de comunicação e orientar melhorias.
    \item Educação em IHC: ensinar designers a pensar nas interfaces como mensagens e não apenas como código e estética.
    \item Design de sistemas críticos: onde falhas de comunicação podem ter alto custo (saúde, transporte, segurança).
\end{itemize}

\section{Diferenças para Outras Abordagens de IHC}
\begin{longtable}{|p{4cm}|p{5cm}|p{5cm}|}
\hline
\textbf{Característica} & \textbf{Engenharia Semiótica} & \textbf{Usabilidade Tradicional} \\
\hline
Foco principal & Comunicação designer–usuário & Eficiência, eficácia e satisfação \\
\hline
Visão da interface & Mensagem/artefato semiótico & Ferramenta funcional \\
\hline
Método de avaliação & Análise de comunicabilidade & Testes de tarefas, métricas quantitativas \\
\hline
Pergunta-chave & ``O usuário entendeu a mensagem?'' & ``O usuário consegue completar a tarefa?'' \\
\hline
\end{longtable}

\section{Benefícios}
\begin{itemize}
    \item Promove interfaces mais autoexplicativas e empáticas.
    \item Considera o contexto cultural e cognitivo do usuário.
    \item Complementa outras métricas de usabilidade.
    \item Dá importância à coerência e clareza do design.
\end{itemize}

\section{Limitações}
\begin{itemize}
    \item Avaliações podem ser mais subjetivas.
    \item Requer formação em semiótica para melhor aplicação.
    \item Não substitui análises quantitativas ou testes com usuários reais.
\end{itemize}

\section{Referências Clássicas}
\begin{itemize}
    \item DE SOUZA, Clarisse Sieckenius. \textit{The Semiotic Engineering of Human–Computer Interaction}. MIT Press, 2005.
    \item PRATES, R.O.; DE SOUZA, C.S. \textit{Métodos de Avaliação de Interfaces}. PUC-Rio.
    \item DE SOUZA, C.S.; LEITÃO, C.F. \textit{Semiotic Engineering Methods for Evaluating User–Designer Communication}. Interacting with Computers, 2009.
\end{itemize}

\end{document}
